% Options for packages loaded elsewhere
% Options for packages loaded elsewhere
\PassOptionsToPackage{unicode}{hyperref}
\PassOptionsToPackage{hyphens}{url}
\PassOptionsToPackage{dvipsnames,svgnames,x11names}{xcolor}
%
\documentclass[
  letterpaper,
  DIV=11,
  numbers=noendperiod]{scrartcl}
\usepackage{xcolor}
\usepackage{amsmath,amssymb}
\setcounter{secnumdepth}{-\maxdimen} % remove section numbering
\usepackage{iftex}
\ifPDFTeX
  \usepackage[T1]{fontenc}
  \usepackage[utf8]{inputenc}
  \usepackage{textcomp} % provide euro and other symbols
\else % if luatex or xetex
  \usepackage{unicode-math} % this also loads fontspec
  \defaultfontfeatures{Scale=MatchLowercase}
  \defaultfontfeatures[\rmfamily]{Ligatures=TeX,Scale=1}
\fi
\usepackage{lmodern}
\ifPDFTeX\else
  % xetex/luatex font selection
\fi
% Use upquote if available, for straight quotes in verbatim environments
\IfFileExists{upquote.sty}{\usepackage{upquote}}{}
\IfFileExists{microtype.sty}{% use microtype if available
  \usepackage[]{microtype}
  \UseMicrotypeSet[protrusion]{basicmath} % disable protrusion for tt fonts
}{}
\makeatletter
\@ifundefined{KOMAClassName}{% if non-KOMA class
  \IfFileExists{parskip.sty}{%
    \usepackage{parskip}
  }{% else
    \setlength{\parindent}{0pt}
    \setlength{\parskip}{6pt plus 2pt minus 1pt}}
}{% if KOMA class
  \KOMAoptions{parskip=half}}
\makeatother
% Make \paragraph and \subparagraph free-standing
\makeatletter
\ifx\paragraph\undefined\else
  \let\oldparagraph\paragraph
  \renewcommand{\paragraph}{
    \@ifstar
      \xxxParagraphStar
      \xxxParagraphNoStar
  }
  \newcommand{\xxxParagraphStar}[1]{\oldparagraph*{#1}\mbox{}}
  \newcommand{\xxxParagraphNoStar}[1]{\oldparagraph{#1}\mbox{}}
\fi
\ifx\subparagraph\undefined\else
  \let\oldsubparagraph\subparagraph
  \renewcommand{\subparagraph}{
    \@ifstar
      \xxxSubParagraphStar
      \xxxSubParagraphNoStar
  }
  \newcommand{\xxxSubParagraphStar}[1]{\oldsubparagraph*{#1}\mbox{}}
  \newcommand{\xxxSubParagraphNoStar}[1]{\oldsubparagraph{#1}\mbox{}}
\fi
\makeatother


\usepackage{longtable,booktabs,array}
\usepackage{calc} % for calculating minipage widths
% Correct order of tables after \paragraph or \subparagraph
\usepackage{etoolbox}
\makeatletter
\patchcmd\longtable{\par}{\if@noskipsec\mbox{}\fi\par}{}{}
\makeatother
% Allow footnotes in longtable head/foot
\IfFileExists{footnotehyper.sty}{\usepackage{footnotehyper}}{\usepackage{footnote}}
\makesavenoteenv{longtable}
\usepackage{graphicx}
\makeatletter
\newsavebox\pandoc@box
\newcommand*\pandocbounded[1]{% scales image to fit in text height/width
  \sbox\pandoc@box{#1}%
  \Gscale@div\@tempa{\textheight}{\dimexpr\ht\pandoc@box+\dp\pandoc@box\relax}%
  \Gscale@div\@tempb{\linewidth}{\wd\pandoc@box}%
  \ifdim\@tempb\p@<\@tempa\p@\let\@tempa\@tempb\fi% select the smaller of both
  \ifdim\@tempa\p@<\p@\scalebox{\@tempa}{\usebox\pandoc@box}%
  \else\usebox{\pandoc@box}%
  \fi%
}
% Set default figure placement to htbp
\def\fps@figure{htbp}
\makeatother


% definitions for citeproc citations
\NewDocumentCommand\citeproctext{}{}
\NewDocumentCommand\citeproc{mm}{%
  \begingroup\def\citeproctext{#2}\cite{#1}\endgroup}
\makeatletter
 % allow citations to break across lines
 \let\@cite@ofmt\@firstofone
 % avoid brackets around text for \cite:
 \def\@biblabel#1{}
 \def\@cite#1#2{{#1\if@tempswa , #2\fi}}
\makeatother
\newlength{\cslhangindent}
\setlength{\cslhangindent}{1.5em}
\newlength{\csllabelwidth}
\setlength{\csllabelwidth}{3em}
\newenvironment{CSLReferences}[2] % #1 hanging-indent, #2 entry-spacing
 {\begin{list}{}{%
  \setlength{\itemindent}{0pt}
  \setlength{\leftmargin}{0pt}
  \setlength{\parsep}{0pt}
  % turn on hanging indent if param 1 is 1
  \ifodd #1
   \setlength{\leftmargin}{\cslhangindent}
   \setlength{\itemindent}{-1\cslhangindent}
  \fi
  % set entry spacing
  \setlength{\itemsep}{#2\baselineskip}}}
 {\end{list}}
\usepackage{calc}
\newcommand{\CSLBlock}[1]{\hfill\break\parbox[t]{\linewidth}{\strut\ignorespaces#1\strut}}
\newcommand{\CSLLeftMargin}[1]{\parbox[t]{\csllabelwidth}{\strut#1\strut}}
\newcommand{\CSLRightInline}[1]{\parbox[t]{\linewidth - \csllabelwidth}{\strut#1\strut}}
\newcommand{\CSLIndent}[1]{\hspace{\cslhangindent}#1}



\setlength{\emergencystretch}{3em} % prevent overfull lines

\providecommand{\tightlist}{%
  \setlength{\itemsep}{0pt}\setlength{\parskip}{0pt}}



 


\usepackage{hyperref}
\usepackage{tikz}
\usepackage{tikz-cd}


\AtBeginDocument{%
  \noindent\textit{This PDF originates from \url{https://sheafofthoughts.org}.}%
  \par\medskip
}

\KOMAoption{captions}{tableheading}
\makeatletter
\@ifpackageloaded{caption}{}{\usepackage{caption}}
\AtBeginDocument{%
\ifdefined\contentsname
  \renewcommand*\contentsname{Table of contents}
\else
  \newcommand\contentsname{Table of contents}
\fi
\ifdefined\listfigurename
  \renewcommand*\listfigurename{List of Figures}
\else
  \newcommand\listfigurename{List of Figures}
\fi
\ifdefined\listtablename
  \renewcommand*\listtablename{List of Tables}
\else
  \newcommand\listtablename{List of Tables}
\fi
\ifdefined\figurename
  \renewcommand*\figurename{Figure}
\else
  \newcommand\figurename{Figure}
\fi
\ifdefined\tablename
  \renewcommand*\tablename{Table}
\else
  \newcommand\tablename{Table}
\fi
}
\@ifpackageloaded{float}{}{\usepackage{float}}
\floatstyle{ruled}
\@ifundefined{c@chapter}{\newfloat{codelisting}{h}{lop}}{\newfloat{codelisting}{h}{lop}[chapter]}
\floatname{codelisting}{Listing}
\newcommand*\listoflistings{\listof{codelisting}{List of Listings}}
\usepackage{amsthm}
\theoremstyle{plain}
\newtheorem{lemma}{Lemma}[section]
\theoremstyle{plain}
\newtheorem{theorem}{Theorem}[section]
\theoremstyle{plain}
\newtheorem{corollary}{Corollary}[section]
\theoremstyle{plain}
\newtheorem{proposition}{Proposition}[section]
\theoremstyle{remark}
\AtBeginDocument{\renewcommand*{\proofname}{Proof}}
\newtheorem*{remark}{Remark}
\newtheorem*{solution}{Solution}
\newtheorem{refremark}{Remark}[section]
\newtheorem{refsolution}{Solution}[section]
\makeatother
\makeatletter
\makeatother
\makeatletter
\@ifpackageloaded{caption}{}{\usepackage{caption}}
\@ifpackageloaded{subcaption}{}{\usepackage{subcaption}}
\makeatother
\usepackage{bookmark}
\IfFileExists{xurl.sty}{\usepackage{xurl}}{} % add URL line breaks if available
\urlstyle{same}
\hypersetup{
  pdftitle={Some interesting consequences of Chebotarev density theorem},
  colorlinks=true,
  linkcolor={blue},
  filecolor={Maroon},
  citecolor={Blue},
  urlcolor={Blue},
  pdfcreator={LaTeX via pandoc}}


\title{Some interesting consequences of Chebotarev density theorem}
\author{}
\date{2025-12-13}
\begin{document}
\maketitle


\section{Introduction}\label{introduction}

The aim of this post is to explain some of the consequences of Cebotarev
density theorem.

First of all, we recall some definition we need, see chapter 13 of
Neukirch (1999).

\textbf{Definition:} Let \(K\) be a number field and \(L/K\) be a Galois
extension with Galois group \(G\). For every \(\sigma \in G\) we define
\(P _{L|K}(\sigma)\) as the set of all unramified prime ideals
\(\mathfrak{p}\) of \(K\) such that there exists a prime ideal
\(\mathfrak{P}|\mathfrak{p}\) of \(L\) satisfying \[
    \sigma=\Big(\frac{L|K}{\mathfrak{P}}\Big)
\] where \(\Big(\frac{L|K}{\mathfrak{P}}\Big)\) is the Frobenius
automorphism of \(\mathfrak{P}\) over \(K\).

Since for all \(\tau\in G\) we have \[
    \Big(\frac{L|K}{\tau \mathfrak{P}}\Big)=\tau \Big(\frac{L|K}{\mathfrak{P}}\Big)\tau ^{-1}
\] the set \(P _{L|K}(\sigma)\) only depends on the conjugacy class \[
\langle \sigma \rangle:=\{\tau\sigma\tau ^{-1} | \tau \in G\}
\] Moreover, if \(\langle\sigma\rangle\neq \langle \tau \rangle\) then
\(P _{L|K}(\sigma)\cap P _{L|K}(\tau)=\emptyset\).

We now want to define what a density is.

\textbf{Definition:} Let \(A \subseteq \mathbb{N}\). Set
\(A(n):=\{1,\dots, n\}\cap A\) and \(a(n):=|A(n)|\). If the limit
exists, we define the \emph{natural density} \(d(A)\) as \[
d(A):=\lim _{n\to \infty}\frac{a(n)}{n}
\]

\subsection{The theorem}\label{the-theorem}

We are now ready to state the theorem

\begin{theorem}[Chebotarev density
theorem]\protect\hypertarget{thm-chebotarev}{}\label{thm-chebotarev}

Let \(K\) be a number field and \(L/K\) be a finite Galois extension
with Galois group \(G\). Then for every \(\sigma \in G\), the set
\(P _{L|K}(\sigma)\) has density (the limit exists), and it is given by
\[
    d(P _{L|K}(\sigma))=\frac{\#\langle \sigma\rangle}{\#G}
\]

\end{theorem}

We will not prove Theorem~\ref{thm-chebotarev}, since the argument is
long and technical. For a complete proof you can look at Neukirch
(1999).

\section{The consequences}\label{the-consequences}

We finally can learn about the corollaries of
Theorem~\ref{thm-chebotarev}.

\subsection{There are infinitely many primes splitting
completely}\label{there-are-infinitely-many-primes-splitting-completely}

\textbf{Definition:} Let \(L/K\) be a Galois extension of number fields
and \(\mathfrak{p}\) be a prime of \(K\). We say that \(\mathfrak{p}\)
splits completely in \(L\) if \[
    f(\mathfrak{p})=e(\mathfrak{p})=1
\] where \(f\) is the inertia index and \(e\) is the ramification index.
We define \(P(L|K)\) as the set of primes of \(K\) splitting completely
in \(L\).

\begin{corollary}[of
Theorem~\ref{thm-chebotarev}]\protect\hypertarget{cor-infinitelyprimessplitting}{}\label{cor-infinitelyprimessplitting}

Let \(K\) be a number field and \(L/K\) be a finite Galois extension.
Then there are infinitely many primes of \(K\) splitting completely in
\(L\).

\end{corollary}

\begin{proof}
Let \(\mathfrak{p}\) be a prime of \(K\). For the same reasoning used in
the proof of Lemma~\ref{lem-primesplitsiff} we have \[
    \mathfrak{p} \text{ splits completely in } L \iff \Big( \frac{L|K}{\mathfrak{P}} \Big)=1_G 
\] Indeed, \(\mathfrak{p}\) splits completely in \(L\) means that it is
unramified and that its inertia index is \(1\). But this means that the
decomposition group \(D(\mathfrak{P}|\mathfrak{p})\) is trivial for
every \(\mathfrak{P}|\mathfrak{p}\). Since the Frobenius automorphism a
generator of the group, \[
    D(\mathfrak{P}|\mathfrak{p})=\{1\}\iff \Big(\frac{L|K}{\mathfrak{P}}\Big)=1_G
\]

Therefore, we get that \(P(L|K)=P _{L|K}(1_G)\). Applying
Theorem~\ref{thm-chebotarev} we obtain \[
    d(P(L|K))=\frac{1}{\# \mathrm{Gal}(L/K)}\neq 0
\] hence the set \(P(L|K)\) must be infinite (otherwise its density
would be \(0\).
\end{proof}

\subsection{Characterization of number fields through splitting of
primes}\label{characterization-of-number-fields-through-splitting-of-primes}

\begin{lemma}[]\protect\hypertarget{lem-primesplitsiff}{}\label{lem-primesplitsiff}

Let \(K\) be a number field, let \(L\) and \(M\) be two finite Galois
extensions of \(K\), and \(N:=LM\) the composite field. For every prime
\(\mathfrak{p}\) of \(K\) unramified in \(N\), pick a prime
\(\mathfrak{P}\) of \(N\) above \(\mathfrak{p}\). Then \[
    \mathfrak{p}\in P(M|K)\iff \Big(\frac{N|K}{\mathfrak{P}}\Big)\in H_M
\]

\end{lemma}

\begin{proof}
Since \(M/K\) is Galois, the restriction map \[
\mathrm{res}_M: G=\mathrm{Gal}(N/K)\to \mathrm{Gal}(M/K)
\] is surjective with kernel \(H_M\). Since \(\mathfrak{p}\) is
unramified, the Frobenius automorphism is well defined and it satisfies:
\[
    \mathrm{res}_M\Big(\Big(\frac{N|K}{\mathfrak{P}}\Big)\Big)=\Big(\frac{M|K}{\mathfrak{P}\cap M}\Big)\in \mathrm{Gal}(M/K)
\] Now observe that \(\mathfrak{p}\) splits completely in \(M\) if and
only if every prime of \(M\) above \(\mathfrak{p}\) has inertia index
\(1\) (since we already know that \(\mathfrak{p}\) is unramified).
Equivalently, (as seen in the proof of
Corollary~\ref{cor-infinitelyprimessplitting}) \[
    \Big(\frac{M|K}{\mathfrak{P}}\Big)=1.
\] This holds if and only if
\(\Big(\frac{N|K}{\mathfrak{P}}\Big)\in \mathrm{ker}(\mathrm{res}_M)=H_M\).
\end{proof}

\begin{proposition}[M.
Bauer]\protect\hypertarget{prp-characterization-of-number-fields}{}\label{prp-characterization-of-number-fields}

Let \(K\) be a number field and let \(L\) and \(M\) be two finite Galois
extensions of \(K\). Then \[
    P(M|K) \subseteq P(L|K) \iff L \subseteq M
\] Therefore, \[
    P(M|K)=P(L|K)\iff L=M
\] In other words, the primes splitting completely determine univocally
the number field.

\end{proposition}

\begin{proof}
If \(L \subseteq M\) then \(P(M|K) \subseteq P(L|K)\) because of the
multiplicativity of inertia and ramification index in towers of
extensions.

Conversely, define the composite field \(N:=LM\) and set \[
G:=\mathrm{Gal}(N/K),\quad H_L:=\mathrm{Gal}(N/L),\quad H_M:=\mathrm{Gal}(N/M).
\]

Assume for the sake of contradiction that \(L\not\subseteq M\). This is
equivalent to \(H_M\not \subseteq H_L\). Choose
\(g\in H_M\smallsetminus H_L\). Since \(H_M\) and \(H_L\) are normal, \[
\langle g\rangle \subseteq H_M,\quad \langle g\rangle \cap H_L=\emptyset
\]

By Theorem~\ref{thm-chebotarev} there are infinitely many primes
\(\mathfrak{p}\) of \(K\) unramified in \(N\) with Frobenius conjugacy
class equal to \(\langle g\rangle\) (if it was finite, the density would
be \(0\)). For such a \(\mathfrak{p}\), pick
\(\mathfrak{P}|\mathfrak{p}\) in \(N\); then \[
    \Big(\frac{N|K}{\mathfrak{P}}\Big)\in \langle g\rangle \subseteq H_M
\]

By Lemma~\ref{lem-primesplitsiff} \[
\mathfrak{p}\in P(M|K)
\] But also \(\Big(\frac{N|K}{\mathfrak{P}}\Big)\notin H_L\), so
\(\mathfrak{p}\notin P(L|K)\). This contradicts
\(P(M|K) \subseteq P(L|K)\).
\end{proof}

\phantomsection\label{refs}
\begin{CSLReferences}{1}{0}
\bibitem[\citeproctext]{ref-Neukirch1999ANT}
Neukirch, Jürgen. 1999. \emph{Algebraic Number Theory}. 1st ed. Vol.
322. Grundlehren Der Mathematischen Wissenschaften. Berlin, Heidelberg:
Springer-Verlag. \url{https://doi.org/10.1007/978-3-662-03983-0}.

\end{CSLReferences}




\end{document}
