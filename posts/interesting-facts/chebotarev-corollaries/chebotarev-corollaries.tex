% Options for packages loaded elsewhere
% Options for packages loaded elsewhere
\PassOptionsToPackage{unicode}{hyperref}
\PassOptionsToPackage{hyphens}{url}
\PassOptionsToPackage{dvipsnames,svgnames,x11names}{xcolor}
%
\documentclass[
  letterpaper,
  DIV=11,
  numbers=noendperiod]{scrartcl}
\usepackage{xcolor}
\usepackage{amsmath,amssymb}
\setcounter{secnumdepth}{-\maxdimen} % remove section numbering
\usepackage{iftex}
\ifPDFTeX
  \usepackage[T1]{fontenc}
  \usepackage[utf8]{inputenc}
  \usepackage{textcomp} % provide euro and other symbols
\else % if luatex or xetex
  \usepackage{unicode-math} % this also loads fontspec
  \defaultfontfeatures{Scale=MatchLowercase}
  \defaultfontfeatures[\rmfamily]{Ligatures=TeX,Scale=1}
\fi
\usepackage{lmodern}
\ifPDFTeX\else
  % xetex/luatex font selection
\fi
% Use upquote if available, for straight quotes in verbatim environments
\IfFileExists{upquote.sty}{\usepackage{upquote}}{}
\IfFileExists{microtype.sty}{% use microtype if available
  \usepackage[]{microtype}
  \UseMicrotypeSet[protrusion]{basicmath} % disable protrusion for tt fonts
}{}
\makeatletter
\@ifundefined{KOMAClassName}{% if non-KOMA class
  \IfFileExists{parskip.sty}{%
    \usepackage{parskip}
  }{% else
    \setlength{\parindent}{0pt}
    \setlength{\parskip}{6pt plus 2pt minus 1pt}}
}{% if KOMA class
  \KOMAoptions{parskip=half}}
\makeatother
% Make \paragraph and \subparagraph free-standing
\makeatletter
\ifx\paragraph\undefined\else
  \let\oldparagraph\paragraph
  \renewcommand{\paragraph}{
    \@ifstar
      \xxxParagraphStar
      \xxxParagraphNoStar
  }
  \newcommand{\xxxParagraphStar}[1]{\oldparagraph*{#1}\mbox{}}
  \newcommand{\xxxParagraphNoStar}[1]{\oldparagraph{#1}\mbox{}}
\fi
\ifx\subparagraph\undefined\else
  \let\oldsubparagraph\subparagraph
  \renewcommand{\subparagraph}{
    \@ifstar
      \xxxSubParagraphStar
      \xxxSubParagraphNoStar
  }
  \newcommand{\xxxSubParagraphStar}[1]{\oldsubparagraph*{#1}\mbox{}}
  \newcommand{\xxxSubParagraphNoStar}[1]{\oldsubparagraph{#1}\mbox{}}
\fi
\makeatother


\usepackage{longtable,booktabs,array}
\usepackage{calc} % for calculating minipage widths
% Correct order of tables after \paragraph or \subparagraph
\usepackage{etoolbox}
\makeatletter
\patchcmd\longtable{\par}{\if@noskipsec\mbox{}\fi\par}{}{}
\makeatother
% Allow footnotes in longtable head/foot
\IfFileExists{footnotehyper.sty}{\usepackage{footnotehyper}}{\usepackage{footnote}}
\makesavenoteenv{longtable}
\usepackage{graphicx}
\makeatletter
\newsavebox\pandoc@box
\newcommand*\pandocbounded[1]{% scales image to fit in text height/width
  \sbox\pandoc@box{#1}%
  \Gscale@div\@tempa{\textheight}{\dimexpr\ht\pandoc@box+\dp\pandoc@box\relax}%
  \Gscale@div\@tempb{\linewidth}{\wd\pandoc@box}%
  \ifdim\@tempb\p@<\@tempa\p@\let\@tempa\@tempb\fi% select the smaller of both
  \ifdim\@tempa\p@<\p@\scalebox{\@tempa}{\usebox\pandoc@box}%
  \else\usebox{\pandoc@box}%
  \fi%
}
% Set default figure placement to htbp
\def\fps@figure{htbp}
\makeatother


% definitions for citeproc citations
\NewDocumentCommand\citeproctext{}{}
\NewDocumentCommand\citeproc{mm}{%
  \begingroup\def\citeproctext{#2}\cite{#1}\endgroup}
\makeatletter
 % allow citations to break across lines
 \let\@cite@ofmt\@firstofone
 % avoid brackets around text for \cite:
 \def\@biblabel#1{}
 \def\@cite#1#2{{#1\if@tempswa , #2\fi}}
\makeatother
\newlength{\cslhangindent}
\setlength{\cslhangindent}{1.5em}
\newlength{\csllabelwidth}
\setlength{\csllabelwidth}{3em}
\newenvironment{CSLReferences}[2] % #1 hanging-indent, #2 entry-spacing
 {\begin{list}{}{%
  \setlength{\itemindent}{0pt}
  \setlength{\leftmargin}{0pt}
  \setlength{\parsep}{0pt}
  % turn on hanging indent if param 1 is 1
  \ifodd #1
   \setlength{\leftmargin}{\cslhangindent}
   \setlength{\itemindent}{-1\cslhangindent}
  \fi
  % set entry spacing
  \setlength{\itemsep}{#2\baselineskip}}}
 {\end{list}}
\usepackage{calc}
\newcommand{\CSLBlock}[1]{\hfill\break\parbox[t]{\linewidth}{\strut\ignorespaces#1\strut}}
\newcommand{\CSLLeftMargin}[1]{\parbox[t]{\csllabelwidth}{\strut#1\strut}}
\newcommand{\CSLRightInline}[1]{\parbox[t]{\linewidth - \csllabelwidth}{\strut#1\strut}}
\newcommand{\CSLIndent}[1]{\hspace{\cslhangindent}#1}



\setlength{\emergencystretch}{3em} % prevent overfull lines

\providecommand{\tightlist}{%
  \setlength{\itemsep}{0pt}\setlength{\parskip}{0pt}}



 


\usepackage{hyperref}
\usepackage{tikz}
\usepackage{tikz-cd}


\AtBeginDocument{%
  \noindent\textit{This PDF originates from \url{https://sheafofthoughts.org}.}%
  \par\medskip
}

\KOMAoption{captions}{tableheading}
\makeatletter
\@ifpackageloaded{caption}{}{\usepackage{caption}}
\AtBeginDocument{%
\ifdefined\contentsname
  \renewcommand*\contentsname{Table of contents}
\else
  \newcommand\contentsname{Table of contents}
\fi
\ifdefined\listfigurename
  \renewcommand*\listfigurename{List of Figures}
\else
  \newcommand\listfigurename{List of Figures}
\fi
\ifdefined\listtablename
  \renewcommand*\listtablename{List of Tables}
\else
  \newcommand\listtablename{List of Tables}
\fi
\ifdefined\figurename
  \renewcommand*\figurename{Figure}
\else
  \newcommand\figurename{Figure}
\fi
\ifdefined\tablename
  \renewcommand*\tablename{Table}
\else
  \newcommand\tablename{Table}
\fi
}
\@ifpackageloaded{float}{}{\usepackage{float}}
\floatstyle{ruled}
\@ifundefined{c@chapter}{\newfloat{codelisting}{h}{lop}}{\newfloat{codelisting}{h}{lop}[chapter]}
\floatname{codelisting}{Listing}
\newcommand*\listoflistings{\listof{codelisting}{List of Listings}}
\usepackage{amsthm}
\theoremstyle{plain}
\newtheorem{theorem}{Theorem}[section]
\theoremstyle{plain}
\newtheorem{lemma}{Lemma}[section]
\theoremstyle{plain}
\newtheorem{corollary}{Corollary}[section]
\theoremstyle{remark}
\AtBeginDocument{\renewcommand*{\proofname}{Proof}}
\newtheorem*{remark}{Remark}
\newtheorem*{solution}{Solution}
\newtheorem{refremark}{Remark}[section]
\newtheorem{refsolution}{Solution}[section]
\makeatother
\makeatletter
\makeatother
\makeatletter
\@ifpackageloaded{caption}{}{\usepackage{caption}}
\@ifpackageloaded{subcaption}{}{\usepackage{subcaption}}
\makeatother
\usepackage{bookmark}
\IfFileExists{xurl.sty}{\usepackage{xurl}}{} % add URL line breaks if available
\urlstyle{same}
\hypersetup{
  pdftitle={Some interesting consequences of Chebotarev density theorem},
  colorlinks=true,
  linkcolor={blue},
  filecolor={Maroon},
  citecolor={Blue},
  urlcolor={Blue},
  pdfcreator={LaTeX via pandoc}}


\title{Some interesting consequences of Chebotarev density theorem}
\author{}
\date{2025-12-13}
\begin{document}
\maketitle


\section{Introduction}\label{introduction}

First of all, we recall some definitions we need; see Chapter VII, §13
of Neukirch (1999).

\textbf{Definition:} Let \(K\) be a number field and let \(L/K\) be a
(finite) Galois extension with Galois group \(G\). For every
\(\sigma \in G\) we define \(P_{L|K}(\sigma)\) as the set of all prime
ideals \(\mathfrak{p}\) of \(K\) that are unramified in \(L\) and such
that there exists a prime ideal \(\mathfrak{P}|\mathfrak{p}\) of \(L\)
satisfying \[
    \sigma=\Big(\frac{L|K}{\mathfrak{P}}\Big),
\] where \(\Big(\frac{L|K}{\mathfrak{P}}\Big)\) is the Frobenius
automorphism of \(\mathfrak{P}\) over \(\mathfrak{p}\).

Since for all \(\tau\in G\) we have \[
    \Big(\frac{L|K}{\tau \mathfrak{P}}\Big)=\tau \Big(\frac{L|K}{\mathfrak{P}}\Big)\tau ^{-1},
\] the set \(P_{L|K}(\sigma)\) only depends on the conjugacy class \[
\langle \sigma \rangle:=\{\tau\sigma\tau ^{-1} \mid \tau \in G\}.
\] Moreover, if \(\langle\sigma\rangle\neq \langle \tau \rangle\) then
\(P_{L|K}(\sigma)\cap P_{L|K}(\tau)=\emptyset\).

We now want to define what a density is (for sets of primes of a number
field).

\textbf{Definition:} Let \(K\) be a number field and let \(S\) be a set
of nonzero prime ideals of \(\mathcal{O}_K\). For \(x\ge 1\) set \[
S(x):=\{\mathfrak{p}\in S \mid N\mathfrak{p}\le x\},\qquad \pi_S(x):=|S(x)|.
\] Let \(\pi_K(x)\) be the number of all nonzero prime ideals
\(\mathfrak{p}\subseteq \mathcal{O}_K\) with \(N\mathfrak{p}\le x\). If
the limit exists, we define the \emph{natural density} \(d(S)\) as \[
d(S):=\lim_{x\to \infty}\frac{\pi_S(x)}{\pi_K(x)}.
\]

\subsection{The theorem}\label{the-theorem}

We are now ready to state the theorem.

\begin{theorem}[Chebotarev density
theorem]\protect\hypertarget{thm-chebotarev}{}\label{thm-chebotarev}

Let \(K\) be a number field and let \(L/K\) be a finite Galois extension
with Galois group \(G\). Then for every \(\sigma \in G\), the set
\(P_{L|K}(\sigma)\) has density (the limit exists), and it is given by
\[
    d(P_{L|K}(\sigma))=\frac{\#\langle \sigma\rangle}{\#G}.
\]

\end{theorem}

We will not prove Theorem~\ref{thm-chebotarev}, since the argument is
long and technical. For a complete proof you can look at Neukirch
(1999).

\section{The consequences}\label{the-consequences}

We can finally look at some corollaries of Theorem~\ref{thm-chebotarev}.

\subsection{There are infinitely many primes splitting
completely}\label{there-are-infinitely-many-primes-splitting-completely}

\textbf{Definition:} Let \(L/K\) be a Galois extension of number fields
and let \(\mathfrak{p}\) be a prime of \(K\). We say that
\(\mathfrak{p}\) splits completely in \(L\) if \[
    e(\mathfrak{p})=f(\mathfrak{p})=1,
\] where \(e(\mathfrak{p})\) is the ramification index and
\(f(\mathfrak{p})\) is the residue degree. We define \(P(L|K)\) as the
set of primes of \(K\) splitting completely in \(L\).

\begin{corollary}[of
Theorem~\ref{thm-chebotarev}]\protect\hypertarget{cor-infinitelyprimessplitting}{}\label{cor-infinitelyprimessplitting}

Let \(K\) be a number field and let \(L/K\) be a finite Galois
extension. Then there are infinitely many primes of \(K\) splitting
completely in \(L\).

\end{corollary}

\begin{proof}
Let \(\mathfrak{p}\) be a prime of \(K\) unramified in \(L\), and let
\(\mathfrak{P}|\mathfrak{p}\) in \(L\). Then \[
    \mathfrak{p} \text{ splits completely in } L \iff \Big( \frac{L|K}{\mathfrak{P}} \Big)=1_G.
\] Indeed, \(\mathfrak{p}\) splits completely means
\(e(\mathfrak{P}|\mathfrak{p})=f(\mathfrak{P}|\mathfrak{p})=1\), hence
the decomposition group \(D(\mathfrak{P}|\mathfrak{p})\) is trivial.
Since \(\mathfrak{p}\) is unramified, the decomposition group is cyclic
and generated by the Frobenius element, so \[
    D(\mathfrak{P}|\mathfrak{p}) \text{ is trivial } \iff \Big(\frac{L|K}{\mathfrak{P}}\Big)=1_G.
\]

Therefore, we get \(P(L|K)=P_{L|K}(1_G)\). Applying
Theorem~\ref{thm-chebotarev} we obtain \[
    d(P(L|K))=\frac{1}{\# \mathrm{Gal}(L/K)}> 0,
\] hence the set \(P(L|K)\) must be infinite (otherwise its density
would be \(0\)).
\end{proof}

\subsection{Characterization of number fields through splitting of
primes}\label{characterization-of-number-fields-through-splitting-of-primes}

\begin{lemma}[]\protect\hypertarget{lem-primesplitsiff}{}\label{lem-primesplitsiff}

Let \(K\) be a number field, let \(L\) and \(M\) be two finite Galois
extensions of \(K\), and \(N:=LM\) the composite field. Set \[
G:=\mathrm{Gal}(N/K),\qquad H_M:=\mathrm{Gal}(N/M).
\] For every prime \(\mathfrak{p}\) of \(K\) unramified in \(N\), pick a
prime \(\mathfrak{P}\) of \(N\) above \(\mathfrak{p}\). Then \[
    \mathfrak{p}\in P(M|K)\iff \Big(\frac{N|K}{\mathfrak{P}}\Big)\in H_M.
\]

\end{lemma}

\begin{proof}
Since \(M/K\) is Galois, the restriction map \[
\mathrm{res}_M: G=\mathrm{Gal}(N/K)\to \mathrm{Gal}(M/K)
\] is surjective with kernel \(H_M\). Since \(\mathfrak{p}\) is
unramified in \(N\), the Frobenius element is well-defined and it
satisfies \[
    \mathrm{res}_M\Big(\Big(\frac{N|K}{\mathfrak{P}}\Big)\Big)=\Big(\frac{M|K}{\mathfrak{P}\cap M}\Big)\in \mathrm{Gal}(M/K).
\]

Now observe that \(\mathfrak{p}\) splits completely in \(M\) if and only
if (equivalently) the Frobenius element in \(\mathrm{Gal}(M/K)\) is
trivial: \[
    \mathfrak{p}\in P(M|K)\iff \Big(\frac{M|K}{\mathfrak{P}\cap M}\Big)=1.
\] This holds if and only if
\(\Big(\frac{N|K}{\mathfrak{P}}\Big)\in \mathrm{ker}(\mathrm{res}_M)=H_M\).
\end{proof}

\begin{corollary}[M. Bauer, corollary of
Theorem~\ref{thm-chebotarev}]\protect\hypertarget{cor-characterization-of-number-fields}{}\label{cor-characterization-of-number-fields}

Let \(K\) be a number field and let \(L\) and \(M\) be two finite Galois
extensions of \(K\). Then \[
    P(M|K) \subseteq P(L|K) \iff L \subseteq M.
\] Therefore, \[
    P(M|K)=P(L|K)\iff L=M.
\] In other words, the primes splitting completely determine uniquely
the number field.

\end{corollary}

\begin{proof}
If \(L \subseteq M\) then \(P(M|K) \subseteq P(L|K)\) because in a tower
the ramification indices and residue degrees are multiplicative.

Conversely, define the composite field \(N:=LM\) and set \[
G:=\mathrm{Gal}(N/K),\quad H_L:=\mathrm{Gal}(N/L),\quad H_M:=\mathrm{Gal}(N/M).
\]

Assume for the sake of contradiction that \(L\not\subseteq M\). This is
equivalent to \(H_M\not \subseteq H_L\). Choose
\(g\in H_M\smallsetminus H_L\). Since \(H_M\) and \(H_L\) are normal, \[
\langle g\rangle \subseteq H_M,\quad \langle g\rangle \cap H_L=\emptyset.
\]

By Theorem~\ref{thm-chebotarev} there are infinitely many primes
\(\mathfrak{p}\) of \(K\) unramified in \(N\) with Frobenius conjugacy
class equal to \(\langle g\rangle\) (if it was finite, the density would
be \(0\)). For such a \(\mathfrak{p}\), pick
\(\mathfrak{P}|\mathfrak{p}\) in \(N\); then \[
    \Big(\frac{N|K}{\mathfrak{P}}\Big)\in \langle g\rangle \subseteq H_M.
\] By Lemma~\ref{lem-primesplitsiff} we get \(\mathfrak{p}\in P(M|K)\).

But also \(\Big(\frac{N|K}{\mathfrak{P}}\Big)\notin H_L\), hence (by the
same lemma applied to \(L\)) we have \(\mathfrak{p}\notin P(L|K)\). This
contradicts \(P(M|K) \subseteq P(L|K)\).

Therefore \(L\subseteq M\).
\end{proof}

\subsection{Dirichlet's Theorem on primes in arithmetic
progressions}\label{dirichlets-theorem-on-primes-in-arithmetic-progressions}

\begin{corollary}[Dirichlet's
Theorem]\protect\hypertarget{cor-dirichlet}{}\label{cor-dirichlet}

Let \(a,n\geq 1\) be integers with \(\gcd(a,n)=1\). Then there are
infinitely many primes such that \(p\equiv a \pmod n\).

\end{corollary}

\begin{proof}
Let \(\zeta\) be a primitive \(n\)-th root of unity, define
\(L:=\mathbb{Q}(\zeta)\), and let \(G:=\mathrm{Gal}(L/\mathbb{Q})\). We
know that \(L/\mathbb{Q}\) is Galois and \[
    G\cong (\mathbb{Z}/n \mathbb{Z})^\times,
\] where the class of \(b\) corresponds to the automorphism sending
\(\zeta\) to \(\zeta^b\).

Let \(p\) be a prime with \(p\nmid n\). Then \(p\) is unramified in
\(L\), and for any prime \(\mathfrak{P}|p\) in \(L\) the Frobenius
automorphism satisfies \[
    \Big(\frac{L|\mathbb{Q}}{\mathfrak{P}}\Big)(\zeta)=\zeta^p.
\] In other words, under the isomorphism
\(G\cong(\mathbb{Z}/n\mathbb{Z})^\times\), the Frobenius element
corresponds to the class of \(p \bmod n\).

Therefore, \[
    p\equiv a \pmod n
    \quad\iff\quad
    \Big(\frac{L|\mathbb{Q}}{\mathfrak{P}}\Big)\ \text{corresponds to } a\in(\mathbb{Z}/n\mathbb{Z})^\times.
\] By Theorem~\ref{thm-chebotarev}, the set of such primes has positive
density, hence it is infinite. (The finitely many primes dividing \(n\)
are irrelevant.)
\end{proof}

\subsection{Condition for linearity of a
polynomial}\label{condition-for-linearity-of-a-polynomial}

\begin{corollary}[]\protect\hypertarget{cor-irrpolylinear}{}\label{cor-irrpolylinear}

Let \(f\in \mathbb{Z}[X]\) be an irreducible polynomial such that for
all but finitely many primes \(p\) the reduction
\(\bar f\in \mathbb{F}_p[X]\) has a root. Then \(\deg(f)=1\).

\end{corollary}

\begin{proof}
Let \(n:=\deg(f)\) and suppose that \(n>1\). Let \(L\) be the splitting
field of \(f\) over \(\mathbb{Q}\) and set
\(G:=\mathrm{Gal}(L/\mathbb{Q})\). Let \(\Omega\) be the set of roots of
\(f\) in \(L\). Since \(f\) is irreducible, the action of \(G\) on
\(\Omega\) is transitive. By Burnside's lemma (orbit-counting), there is
some \(\sigma\in G\) with no fixed points in \(\Omega\).

Let \(S\) be the finite set of rational primes \(p\) such that \(p\)
divides the discriminant of \(f\) or the leading coefficient of \(f\).
For every \(p\notin S\), the reduction \((f\bmod p)\) is separable, and
\(p\) is unramified in the splitting field \(L\). Moreover, the
factorization type of \((f\bmod p)\) corresponds to the cycle structure
of \(\Big(\frac{L|\mathbb{Q}}{\mathfrak{P}}\Big)\) acting on \(\Omega\),
where \(\mathfrak{P}|p\). In particular, \((f\bmod p)\) has a root in
\(\mathbb{F}_p\) if and only if
\(\Big(\frac{L|\mathbb{Q}}{\mathfrak{P}}\Big)\) has a fixed point in
\(\Omega\).

By Theorem~\ref{thm-chebotarev}, the set \(P_{L|\mathbb{Q}}(\sigma)\) is
infinite (it has positive density). Since \(S\) is finite, there are
infinitely many primes \(p\notin S\) such that \[
    \Big(\frac{L|\mathbb{Q}}{\mathfrak{P}}\Big)\in \langle\sigma\rangle,
\] where \(\mathfrak{P}|p\). For all such primes the Frobenius
automorphism has no fixed points, thus the reduction of \(f\) has no
roots modulo these primes. This is a contradiction, since we assumed
that \((f\bmod p)\) has a root for all but finitely many primes.
\end{proof}

\subsection{\texorpdfstring{How many
\(\mathbb{F}_p\)-points?}{How many \textbackslash mathbb\{F\}\_p-points?}}\label{how-many-mathbbf_p-points}

\begin{corollary}[]\protect\hypertarget{cor-howmanyFppoints}{}\label{cor-howmanyFppoints}

Let \(A\) be a \(\mathbb{Z}\)-algebra of finite type such that there
exists a ring map \(f:A\to \mathbb{C}\). Then for infinitely many primes
\(p\) there is a ring morphism \(A\to \mathbb{F}_p\).

\end{corollary}

What we are saying is that if you consider a set of polynomials in
\(\mathbb{Z}[X_1,\dots , X_n]\) (the polynomials defining \(A\) as a
quotient of \(\mathbb{Z}[X_1,\dots, X_n]\)), then if they have a common
solution in \(\mathbb{C}\) (i.e.~there is a map \(A\to \mathbb{C}\)),
they have a common solution modulo \(p\), for infinitely many primes
\(p\) (i.e.~a map \(A\to \mathbb{F}_p\)).

\begin{proof}
Since we have a map \(f:A\to \mathbb{C}\), we get
\(A\otimes _ \mathbb{Z}\mathbb{C}\neq 0\), which is equivalent to
\(A\otimes _\mathbb{Z}\mathbb{Q}\neq 0\) (because
\(\mathbb{Q}\to \mathbb{C}\) is faithfully flat). In particular,
\(A\otimes _\mathbb{Z}\mathbb{Q}\) has a maximal ideal \(\mathfrak{m}\).
Let \[
L := (A\otimes_\mathbb{Z}\mathbb{Q})/\mathfrak{m}.
\] By Zariski's lemma, \(L\) is a finite extension of \(\mathbb{Q}\),
hence a number field.

Consider now the composition map
\(\phi: A\to A\otimes_\mathbb{Z}\mathbb{Q}\to L\). Let \(a_1,\dots,a_r\)
be generators of \(A\) as a \(\mathbb{Z}\)-algebra. Choose
\(N\in \mathbb{Z}_{>0}\) such that \(N\cdot \phi(a_i)\) is an algebraic
integer for every \(i=1,\dots, r\). Then
\(\phi(a_i)\in \mathcal{O}_L[1/N]\) for all \(i\), hence \(\phi\) lands
in \(\mathcal{O}_L[1/N]\), i.e.~we have a map \[
\phi: A\to \mathcal{O}_L[1/N].
\] Equivalently, it factorizes through a map \[
\bar \phi: A[1/N]\to \mathcal{O}_L[1/N].
\]

Fix a rational prime \(p\) not dividing \(N\). Any prime ideal
\(\mathfrak{P}\) of \(L\) dividing \(p\) gives the reduction map \[
    \mathcal{O}_L[1/N]\to \mathcal{O}_L/\mathfrak{P}.
\] Moreover, if the residue degree satisfies \(f(\mathfrak{P}|p)=1\),
then \(\mathcal{O}_L/\mathfrak{P}\cong \mathbb{F}_p\), hence we get a
map \[
    A[1/N]\to \mathcal{O}_L[1/N]\to \mathcal{O}_L/\mathfrak{P}\cong \mathbb{F}_p.
\] Since \(p\nmid N\), this is the same as a map \(A\to \mathbb{F}_p\).

Let \(M\) be the Galois closure of \(L\) over \(\mathbb{Q}\). By
Corollary~\ref{cor-infinitelyprimessplitting} there are infinitely many
primes \(p\) splitting completely in \(M\). Discarding the finitely many
primes dividing \(N\), we can assume also \(p\nmid N\).

Let \(p\) be such a prime. Choose a prime \(\mathfrak{Q}\) of \(M\)
above \(p\) and set \(\mathfrak{P}:=\mathfrak{Q}\cap \mathcal{O}_L\).
Since \(p\) splits completely in \(M\), we have \(f(\mathfrak{Q}|p)=1\).
By multiplicativity of residue degrees in towers, \[
f(\mathfrak{Q}|p)=f(\mathfrak{Q}|\mathfrak{P})\cdot f(\mathfrak{P}|p)=1,
\] hence \(f(\mathfrak{P}|p)=1\) and therefore
\(\mathcal{O}_L/\mathfrak{P}\cong \mathbb{F}_p\). This gives a map
\(A\to \mathbb{F}_p\) for infinitely many primes \(p\).
\end{proof}

\section{Conclusion}\label{conclusion}

In this post we saw how the Chebotarev density theorem can be used as a
very efficient ``machine'' to produce arithmetic information.

Starting from the simple observation that Frobenius elements encode
splitting behaviour, we deduced that totally split primes always exist
in abundance (in fact with positive density), and we used this to prove
a rigidity statement: in the Galois case, knowing which primes split
completely already determines the extension.

We also proved Dirichlet's theorem on primes in arithmetic progressions
by applying Chebotarev to cyclotomic fields, and we saw a typical
application to polynomials: if an irreducible polynomial has a root
modulo almost every prime, then it must be linear.

All these results share the same pattern: translate a question into a
condition on Frobenius conjugacy classes, then use Chebotarev to
conclude that the relevant primes not only exist, but have a precise
density.

\phantomsection\label{refs}
\begin{CSLReferences}{1}{0}
\bibitem[\citeproctext]{ref-Neukirch1999ANT}
Neukirch, Jürgen. 1999. \emph{Algebraic Number Theory}. 1st ed. Vol.
322. Grundlehren Der Mathematischen Wissenschaften. Berlin, Heidelberg:
Springer-Verlag. \url{https://doi.org/10.1007/978-3-662-03983-0}.

\end{CSLReferences}




\end{document}
