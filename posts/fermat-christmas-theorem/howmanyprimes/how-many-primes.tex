% Options for packages loaded elsewhere
% Options for packages loaded elsewhere
\PassOptionsToPackage{unicode}{hyperref}
\PassOptionsToPackage{hyphens}{url}
\PassOptionsToPackage{dvipsnames,svgnames,x11names}{xcolor}
%
\documentclass[
  letterpaper,
  DIV=11,
  numbers=noendperiod]{scrartcl}
\usepackage{xcolor}
\usepackage{amsmath,amssymb}
\setcounter{secnumdepth}{-\maxdimen} % remove section numbering
\usepackage{iftex}
\ifPDFTeX
  \usepackage[T1]{fontenc}
  \usepackage[utf8]{inputenc}
  \usepackage{textcomp} % provide euro and other symbols
\else % if luatex or xetex
  \usepackage{unicode-math} % this also loads fontspec
  \defaultfontfeatures{Scale=MatchLowercase}
  \defaultfontfeatures[\rmfamily]{Ligatures=TeX,Scale=1}
\fi
\usepackage{lmodern}
\ifPDFTeX\else
  % xetex/luatex font selection
\fi
% Use upquote if available, for straight quotes in verbatim environments
\IfFileExists{upquote.sty}{\usepackage{upquote}}{}
\IfFileExists{microtype.sty}{% use microtype if available
  \usepackage[]{microtype}
  \UseMicrotypeSet[protrusion]{basicmath} % disable protrusion for tt fonts
}{}
\makeatletter
\@ifundefined{KOMAClassName}{% if non-KOMA class
  \IfFileExists{parskip.sty}{%
    \usepackage{parskip}
  }{% else
    \setlength{\parindent}{0pt}
    \setlength{\parskip}{6pt plus 2pt minus 1pt}}
}{% if KOMA class
  \KOMAoptions{parskip=half}}
\makeatother
% Make \paragraph and \subparagraph free-standing
\makeatletter
\ifx\paragraph\undefined\else
  \let\oldparagraph\paragraph
  \renewcommand{\paragraph}{
    \@ifstar
      \xxxParagraphStar
      \xxxParagraphNoStar
  }
  \newcommand{\xxxParagraphStar}[1]{\oldparagraph*{#1}\mbox{}}
  \newcommand{\xxxParagraphNoStar}[1]{\oldparagraph{#1}\mbox{}}
\fi
\ifx\subparagraph\undefined\else
  \let\oldsubparagraph\subparagraph
  \renewcommand{\subparagraph}{
    \@ifstar
      \xxxSubParagraphStar
      \xxxSubParagraphNoStar
  }
  \newcommand{\xxxSubParagraphStar}[1]{\oldsubparagraph*{#1}\mbox{}}
  \newcommand{\xxxSubParagraphNoStar}[1]{\oldsubparagraph{#1}\mbox{}}
\fi
\makeatother


\usepackage{longtable,booktabs,array}
\usepackage{calc} % for calculating minipage widths
% Correct order of tables after \paragraph or \subparagraph
\usepackage{etoolbox}
\makeatletter
\patchcmd\longtable{\par}{\if@noskipsec\mbox{}\fi\par}{}{}
\makeatother
% Allow footnotes in longtable head/foot
\IfFileExists{footnotehyper.sty}{\usepackage{footnotehyper}}{\usepackage{footnote}}
\makesavenoteenv{longtable}
\usepackage{graphicx}
\makeatletter
\newsavebox\pandoc@box
\newcommand*\pandocbounded[1]{% scales image to fit in text height/width
  \sbox\pandoc@box{#1}%
  \Gscale@div\@tempa{\textheight}{\dimexpr\ht\pandoc@box+\dp\pandoc@box\relax}%
  \Gscale@div\@tempb{\linewidth}{\wd\pandoc@box}%
  \ifdim\@tempb\p@<\@tempa\p@\let\@tempa\@tempb\fi% select the smaller of both
  \ifdim\@tempa\p@<\p@\scalebox{\@tempa}{\usebox\pandoc@box}%
  \else\usebox{\pandoc@box}%
  \fi%
}
% Set default figure placement to htbp
\def\fps@figure{htbp}
\makeatother





\setlength{\emergencystretch}{3em} % prevent overfull lines

\providecommand{\tightlist}{%
  \setlength{\itemsep}{0pt}\setlength{\parskip}{0pt}}



 


\usepackage{hyperref}

\AtBeginDocument{%
  \noindent\textit{This PDF originates from \url{https://sheafofthoughts.org}.}%
  \par\medskip
}

\KOMAoption{captions}{tableheading}
\makeatletter
\@ifpackageloaded{caption}{}{\usepackage{caption}}
\AtBeginDocument{%
\ifdefined\contentsname
  \renewcommand*\contentsname{Table of contents}
\else
  \newcommand\contentsname{Table of contents}
\fi
\ifdefined\listfigurename
  \renewcommand*\listfigurename{List of Figures}
\else
  \newcommand\listfigurename{List of Figures}
\fi
\ifdefined\listtablename
  \renewcommand*\listtablename{List of Tables}
\else
  \newcommand\listtablename{List of Tables}
\fi
\ifdefined\figurename
  \renewcommand*\figurename{Figure}
\else
  \newcommand\figurename{Figure}
\fi
\ifdefined\tablename
  \renewcommand*\tablename{Table}
\else
  \newcommand\tablename{Table}
\fi
}
\@ifpackageloaded{float}{}{\usepackage{float}}
\floatstyle{ruled}
\@ifundefined{c@chapter}{\newfloat{codelisting}{h}{lop}}{\newfloat{codelisting}{h}{lop}[chapter]}
\floatname{codelisting}{Listing}
\newcommand*\listoflistings{\listof{codelisting}{List of Listings}}
\usepackage{amsthm}
\theoremstyle{plain}
\newtheorem{theorem}{Theorem}[section]
\theoremstyle{remark}
\AtBeginDocument{\renewcommand*{\proofname}{Proof}}
\newtheorem*{remark}{Remark}
\newtheorem*{solution}{Solution}
\newtheorem{refremark}{Remark}[section]
\newtheorem{refsolution}{Solution}[section]
\makeatother
\makeatletter
\makeatother
\makeatletter
\@ifpackageloaded{caption}{}{\usepackage{caption}}
\@ifpackageloaded{subcaption}{}{\usepackage{subcaption}}
\makeatother
\usepackage{bookmark}
\IfFileExists{xurl.sty}{\usepackage{xurl}}{} % add URL line breaks if available
\urlstyle{same}
\hypersetup{
  pdftitle={How many primes are there?},
  colorlinks=true,
  linkcolor={blue},
  filecolor={Maroon},
  citecolor={Blue},
  urlcolor={Blue},
  pdfcreator={LaTeX via pandoc}}


\title{How many primes are there?}
\author{}
\date{2025-12-08}
\begin{document}
\maketitle


\section{Introduction}\label{introduction}

In the
\href{https://sheafofthoughts.org/posts/2025-11-29-fct-introduction/}{previous
post} we asked some questions about prime numbers such as:

\begin{itemize}
\tightlist
\item
  Which numbers could be written as the product of primes?
\item
  How many primes are there?
\end{itemize}

The goal of this post is to give a precise answer to these two
questions.

The first person to rigorously answer these questions was the Greek
mathematician \href{https://en.wikipedia.org/wiki/Euclid}{Euclid}, and
we will roughly follow his reasoning.

\subsection{What a prime is?}\label{what-a-prime-is}

Let's start with a short recap. In school we learned that a positive
integer is prime if its only divisors are \(1\) and itself (\(1\) is not
considered prime). So we have that \(2,3,5,\dots\) are the first few
primes.

Consider the prime \(2\). Note that every time \(2\) divides a product
of two integers \(a\cdot b\), it divides one of the two factors
(i.e.~either \(2\) divides \(a\) or \(2\) divides \(b\)). Try a few
examples to convince yourself. For instance, \(2\) divides the product
\(4\cdot 5=20\) and indeed \(2\) divides \(4\). You can try any product
and the property will hold. There is nothing special about \(2\);
indeed, every prime number satisfies this (i.e.~if a prime number \(p\)
divides a product \(a\cdot b\) then either \(p\) divides \(a\) or \(p\)
divides \(b\)). Note that this property characterises primes. For
example, observe that \(6\) (which isn't prime) divides the product
\(3\cdot 4=12\), but \(6\) divides neither \(3\) nor \(4\).

Since this property is important, we define prime numbers as follows:

\textbf{Definition:} An integer \(p\) is \emph{prime} if whenever \(p\)
divides a product \(a\cdot b\), then \(p\) divides \(a\) or \(p\)
divides \(b\).

We will instead use the term \emph{irreducible} for the other definition
we know:

\textbf{Definition:} An integer \(q\) is \emph{irreducible} if its only
divisors are itself,\(1\), and \(-1\).

We include \(-1\) because we allow negative integers; for instance
\(-3\) is irreducible since its only divisors are \(-1\), \(1\), and
\(-3\).

For integers, a number is irreducible if and only if it is prime, so the
two definitions agree. This is why we often use the irreducible
definition for primes.

\begin{center}\rule{0.5\linewidth}{0.5pt}\end{center}

\section{Back to school: the Euclidean
division}\label{back-to-school-the-euclidean-division}

We know how division of integers works. For example, to divide \(7\) by
\(2\) we count how many times \(2\) can be added without exceeding
\(7\). We see that \(2+2+2=6\), so the quotient is \(3\) and the
remainder is \(1\). We write in symbols

\[
7=2\cdot 3+1
\]

Another example: dividing \(17\) by \(15\) yields quotient \(1\) and
remainder \(2\), i.e.

\[
17=15\cdot 1+2
\]

As a final example, suppose we want to divide \(3\) by \(12\). Since
\(12\) is too big, the quotient is \(0\) and the remainder is \(3\),
i.e.

\[
3=12\cdot 0+3
\]

We learned that we can always perform division with remainder: for
positive integers \(n,m\) we can write

\[
n=m\cdot q+r
\]

where \(q\) is the quotient and \(r\) is the remainder. We are ready to
state this as a theorem:

\begin{theorem}[]\protect\hypertarget{thm-euclideandivision}{}\label{thm-euclideandivision}

Euclidean division.\\
For every two integers \(n\geq 0\) and \(m>0\) (we cannot divide by
\(0\)), there exist integers \(q,r\geq 0\) such that

\[
n=m\cdot q+r
\]

where \(r<m\). Moreover \(q\) and \(r\) are uniquely determined.

\end{theorem}

Okay, pause for a moment: this theorem seems more complicated than it
should be. First, it should be clear that we can write our \(n\) as

\[
n=m\cdot q+r
\]

for some \(q\) and \(r\). Why must \(r\) be strictly less than \(m\)?
Because if it were not, we could divide further. For example suppose we
write

\[
15=4\cdot 2+6
\]

This is not admissible, since \(6=4+2\); we can still divide by \(4\) to
get

\[
15=4\cdot 3+2
\]

Now, since \(2\) is smaller than \(4\), we cannot divide it again, so we
are done and what we get is the division.

The other question is: what does it mean that \(q\) and \(r\) are
uniquely determined? It means there is only one way to write \(n\) as
\(m\cdot q+r\), with \(r<m\). Take a pen and paper and try to convince
yourself that this writing is unique by doing some example.

\begin{center}\rule{0.5\linewidth}{0.5pt}\end{center}

\section{The fundamental theorem of
arithmetic}\label{the-fundamental-theorem-of-arithmetic}

Now we want to answer the question of which numbers could be written as
a product of primes. The following theorem gives us the answer.

\begin{theorem}[]\protect\hypertarget{thm-fundthrmarithmetic}{}\label{thm-fundthrmarithmetic}

Fundamental theorem of arithmetic.\\
Every integer \(n\) strictly bigger than one (i.e.~\(n>1\)) can be
written \emph{uniquely} as the product of prime numbers.

\end{theorem}

This helps us understand why primes are so important. In this sense they
are the basic building blocks of the integers. We can always decompose
an integer as the product of primes and this is done uniquely.

Let's give some examples to clarify the theorem. Consider \(15\). Its
prime divisors are \(3\) and \(5\) so \(15=3\cdot 5\). Another example
could be \(4\); it is divided only by \(2\), so \(4=2^2\).

How do we factor a number in practice? Pick an integer, for example
\(36\). We look at the first few primes

\[
2,3,5,7,11,\dots
\]

and test the divisibility by successive primes:

\begin{itemize}
\tightlist
\item
  Is \(36\) divisible by \(2\)? Yes: \(36=2\cdot 18\).
\item
  Is \(18\) divisible by \(2\)? Yes: \(18=2\cdot 9\).
\item
  Is \(9\) divisible by \(2\)? No: we move on to the next prime.
\item
  Is \(9\) divisible by \(3\)? Yes: \(9=3\cdot 3\).
\end{itemize}

So we get

\[
36=2\cdot 18=2\cdot (2\cdot 9)=2\cdot (2\cdot (3\cdot 3))=2^2\cdot 3^2
\]

By this process we can factor any positive integer. For negative
integers, include a leading minus sign; e.g.~\(-36=-(2^2\cdot 3^2)\).

Now consider \(7\). Since it is prime, it can't be factorized further.
Therefore, we will say that its factorization is just itself, i.e.~given
by

\[
7=7
\]

where on the right hand side we are writing the factors (just \(7\)) of
the factorization.

We want to convince ourselves that a factorization always exists, so we
dive into its proof.

\begin{proof}
Existence of the prime factorization.\\
Suppose some positive integers could not be factorized. Take the
smallest such integer and call it \(m\). The integer \(m\) cannot be a
prime number, since if it were its factorization would be \(m=m\) (as we
did in the case of \(7\)). Since \(m\) is not prime, there must be some
number \(s\) other than \(m\) and \(1\) which divides it. We can
therefore write \(m=s\cdot t\) for some other integer \(t\). But now
observe that \(s\) and \(t\) are both smaller than \(m\). Since \(m\)
was the smallest such number that could not be factorized, both \(s\)
and \(t\) can be factorized into prime factors. In particular we get
that

\[
m=s\cdot t=(\text{prime factorization of }s)\cdot (\text{prime factorization of }t)
\]

This yields a factorization of \(m\). For example, \(144=4\cdot 36\);
factoring gives \(4=2^2\) and \(36=2^2\cdot 3^2\), hence

\[
144=4\cdot 36=(2^2)\cdot (2^2\cdot 3^2)=2^4\cdot 3^2
\]

Therefore, we proved that the smallest number which couldn't be
factorized can actually be factorized. Hence, we get the claim.
\end{proof}

Moreover, the theorem tells us that the factorization is unique: you
can't write an integer (\(>1\)) as a product of primes in two different
ways. Try some examples to convince yourself.

\begin{center}\rule{0.5\linewidth}{0.5pt}\end{center}

\section{How many primes are there?}\label{how-many-primes-are-there}

Finally, we are ready to answer this question. You may have guessed the
answer, but I really want to give a neat argument that will convince you
it is true.

There exist infinitely many prime numbers.

\begin{proof}
Euclid's proof.\\
Suppose there were only finitely many prime numbers, call them
\(p_1,\dots,p_n\). Define \(m\) as the product of all these primes plus
\(1\), i.e.

\[
m=p_1\cdot p_2\cdots p_n+1
\]

Observe that \(m\) is not divisible by any of the primes
\(p_1,\dots,p_n\). To convince ourselves, take \(p_1\) and write

\[
m=p_1\cdot (p_2\cdots p_n)+1
\]

Define \(q:=(p_2\cdots p_n)\). We are saying

\[
m=p_1\cdot (p_2\cdots p_n)+1=p_1\cdot q+1
\]

Thus, \(m\) has remainder \(1\) when divided by \(p_1\), so \(p_1\) does
not divide \(m\) (if it did the remainder would be \(0\)). Nothing is
special about \(p_1\): we can repeat the same argument for all the other
primes \(p_2,\dots,p_n\).

So we are convinced that no prime in the list \(p_1,\dots,p_n\) divides
\(m\).

But we said that any integer number can be written as the product of
primes. However, since \(m\) is not divisible by any prime in the list
\(p_1,\dots,p_n\), it means that the list does not contain all of the
primes.

Hence, whenever we suppose a finite list \(p_1,\dots,p_n\) contains all
primes, we can construct a prime not in the list; thus any finite list
is incomplete.

We conclude that there must be infinitely many primes.
\end{proof}

\begin{center}\rule{0.5\linewidth}{0.5pt}\end{center}

\section{Conclusion}\label{conclusion}

To prove there are \emph{infinitely many} primes we used a common
strategy. We understood that proving the statement as it appears is very
difficult. How can you prove there are infinitely many primes? The
natural idea is: suppose there were only finitely many, and derive a
contradiction. This is what we call a \emph{proof by contradiction} (you
prove your claim by proving that its negation is false).

There is another cool idea we used today. When we proved
Theorem~\ref{thm-fundthrmarithmetic} we had the idea of supposing that
there were some numbers not satisfying our statement. Then we took the
minimum among these numbers and proved that the minimum actually
satisfied our claim. This kind of argument is often called \emph{proof
by minimal counterexample} (a form of induction), and it is also common.

Think about these two kinds of proofs and how we used them in the
arguments. If you'd like to practice, try proving uniqueness of prime
factorization (proceed by contradiction and use the two definitions of
prime introduced above).

\begin{center}\rule{0.5\linewidth}{0.5pt}\end{center}

In any case, see you next week!

\begin{center}\rule{0.5\linewidth}{0.5pt}\end{center}




\end{document}
